\subsection{Drift}
De driftcontroller is een meer gewaagde versie van de speedcontroller, waarbij 
er drie algemene aanpassingen zijn gebeurd. Onderstaande beschrijven we onze 
aanpassingen tegenover de hierboven besproken speedcontroller.

Ten eerste hebben we het sturen bruusker gemaakt. Hiervoor hebben we zowel de
lidmaatschapsfuncties van de termen voor \texttt{steering} iets meer richting de extremen gezet.
We hebben ook de steering regels aangepast zoals in onderstaande regelblokken te zien
zijn. In plaats van gewoon naar links te sturen in een gewone linkse bocht
oversturen we eigenlijk. Idem voor rechts. Deze toevoegingen zorgen ervoor dat
we eigenlijk té veel insturen in een bocht, waardoor de achterkant van de auto
zal uitbreken en we beginnen driften.

Dit driften moeten we natuurlijk ook tegengaan om de auto na het oversturen in
een bocht te stabiliseren. Hiervoor voegen we twee nieuwe regels toe, regel 7 en
8 in het \texttt{steering} ruleblock. Wanneer we in een bocht aan het driften
zijn zal de \texttt{corner} variabele uit deze bocht willen sturen, wat
automatisch al betekent dat de auto zal tegensturen. Dus, als we aan het driften
zijn, en de zij-sensoren zien een niet te scherpe bocht naar links/rechts dan
kunnen we concluderen dat we net hebben overstuurd in een bocht naar
rechts/links. Om dit te corrigeren sturen we in dit geval dus scherp in de bocht
die gezien wordt om het oversturen in de originele bocht tegen te gaan.

Een tweede algemene aanpassing die we hebben gemaakt is het weghalen van 
braking-rules. Om een deftige drift te kunnen uitvoeren moeten we met voldoende 
snelheid in de bocht gaan, anders breekt de achterkant van de auto niet uit. We 
willen ook niet extra gaan remmen tijdens het driften, dus ook deze regels 
halen we weg. 

Als laatste willen we uit op het eind van een drift ook extra gas geven om snel 
weg te rijden (en om wat extra rook te ontwikkelen), dus ook hiervoor is een 
extra regel toegevoegd.

De lidmaatschapsfuncties van deze controllers is te vinden in de rechterkolom 
van figuren \ref{fig:lidfties_in} en \ref{fig:lidfties_out}.

\begin{lstlisting}
RULEBLOCK steering
  RULE 1 : IF corner IS sharp_left_corner THEN steering IS sharp_left;
  RULE 2 : IF corner IS left_corner THEN steering IS sharp_left;
  RULE 3 : IF corner IS mild_left_corner THEN steering IS mild_left;
  RULE 4 : IF corner IS mild_right_corner THEN steering IS mild_right;
  RULE 5 : IF corner IS right_corner THEN steering IS sharp_right;
  RULE 6 : IF corner IS sharp_right_corner THEN steering IS sharp_right;

  RULE 7 : IF lateralVelocity IS drifting AND (corner IS mild_right_corner OR 
  corner IS right_corner) THEN steering IS sharp_right;
  RULE 8 : IF lateralVelocity IS drifting AND (corner IS mild_left_corner  OR 
  corner IS left_corner)  THEN steering IS sharp_left;
END_RULEBLOCK

RULEBLOCK speedup
  RULE 1 : IF carSpeedKph IS slow AND frontSensorDistance IS near THEN 
  acceleration IS fastly;
  RULE 2 : IF carSpeedKph IS slow AND frontSensorDistance IS far THEN 
  acceleration IS fastly;
  RULE 3 : IF carSpeedKph IS normal AND frontSensorDistance IS far THEN 
  acceleration IS fastly;

  RULE 4 : IF carSpeedKph IS fast AND lateralVelocity IS minimal AND 
  frontSensorDistance IS very_far THEN acceleration IS fastly;
  RULE 5 : IF carSpeedKph IS hyper_fast AND frontSensorDistance IS NOT very_far 
  THEN acceleration IS slowly;

  RULE 6 : IF lateralVelocity IS drifting THEN acceleration IS fastly;
END_RULEBLOCK

RULEBLOCK brake
  RULE 4 : IF frontSensorDistance IS NOT far AND carSpeedKph IS fast THEN brake 
  IS slowly;
  RULE 5 : IF frontSensorDistance IS NOT very_far AND carSpeedKph IS very_fast 
  THEN brake IS normal;
  RULE 6 : IF carSpeedKph IS fast THEN brake IS normal;
  RULE 7 : IF carSpeedKph IS very_fast AND frontSensorDistance IS NOT very_far 
  THEN brake IS fastly;
END_RULEBLOCK

RULEBLOCK scanangle
  RULE 1: IF frontSensorDistance is near THEN scanangle IS wide;
  RULE 2: IF carSpeedKph IS very_fast THEN scanangle IS narrow;
  RULE 3: IF carSpeedKph IS hyper_fast THEN scanangle IS very_narrow;
END_RULEBLOCK
\end{lstlisting}